% Compile by directly running 
%   pdflatex Slurm_Intro.tex 
%
%
%
%
%




\documentclass{beamer}
\usetheme{Copenhagen}
\usepackage[utf8]{inputenc}


%\usepackage{graphicx}
%\usepackage{subfigure}
%\usepackage{multimedia}
\usepackage{times}  % fonts are up to you
\usepackage{graphics}
\usepackage{amsmath}
%\usepackage{media9}
\usepackage{hyperref}
\usepackage{psfrag}
\usepackage{pdfpages}
\usepackage{listings}


\setbeamertemplate{bibliography item}[text]
\newcommand{\customcite}[1]{\citeauthor{#1}, \citeyear{#1}}
\newcommand\smallFont{\fontsize{8}{7.2}\selectfont}   %Change font size.
\newcommand\mCite[1]{[\cite{#1}, \citetitle{#1}]}  %Prints name and title
\newcommand\FrameText[1]{
\begin{textblock*}{\paperwidth}(0pt,\textheight)
	\vspace{1.0cm}
    \raggedleft \smallFont #1
\end{textblock*}}

%Get rid of ugly copenhagen default symbol for enumerate
\setbeamertemplate{enumerate items}[default]   


% Create code text
% https://tex.stackexchange.com/questions/65291/code-snippet-in-text
\definecolor{codegray}{gray}{0.9}
\newcommand{\code}[1]{\colorbox{codegray}{\texttt{#1}}}
\hypersetup{colorlinks=true,urlcolor=blue, linktocpage}  % Makes red / pink?



%Information to be included in the title page:
\title{Bash Environment and Managing Software}
\author{Ali Snedden}
\institute{Nationwide Children's Hospital}
\date{August 24, 2021}
 
 
 
\begin{document}
 
\frame{\titlepage}




\begin{frame}
\frametitle{How to Connect}
Windows:
\begin{itemize}
    \item Open PuTTY
    \item Window Session $\Rightarrow$ Host Name field : username@r1pl-hpcf-log01
    \item Click ``Open" to log in.
    \item Enter password
\end{itemize}

Mac:
\begin{itemize}
    \item Open Terminal (Finder $\Rightarrow$ Utilities $\Rightarrow$ Terminal)
    \item \code{ssh -X username@r1pl-hpcf-log01}
\end{itemize}
\end{frame}



\begin{frame}
\frametitle{What is Bash?}
\begin{itemize}
    \item User interface for the Linux operating system
    \pause
    \item I.e. the 'language' used at the command line
    \pause
    \item Interactive Shell on Franklin
    \pause
    \item There are 'other' shells available
    \begin{enumerate}
        \item Korn Shell (\code{ksh}), C Shell (\code{csh}), Z shell (\code{zsh})
    \end{enumerate}
\end{itemize}
\end{frame}


\begin{frame}
\frametitle{What is Bash?}
As a programming language, it has :
\begin{itemize}
    \item Variables 
    \pause
    \item Conditionals (i.e. \code{if/else})
    \pause
    \item Loops 
    \pause
    \item Functions (ignored in this presentation)
\end{itemize}
\end{frame}


\begin{frame}
\frametitle{What is Bash?}
Criticism : 
\begin{itemize}
    \item Syntax is ugly and clunky
    \pause
    \item Doesn't support floating point arithmetic (e.g. \code{VAR = \$(expr 4.1 + 2.3)} )
    \pause
    \item Doesn't support real error handling
    \pause
    \item For complex tasks use a fully functional language (e.g. Python, Perl)
    \pause
    \item Very sensitive to white space and quotes
    \pause
    \item I find myself heavily relying on Google to get things right
\end{itemize}
\end{frame}



\begin{frame}[fragile]
\frametitle{Variables}
Reference :
\begin{itemize}
    \item \href{https://tldp.org/LDP/abs/html/parameter-substitution.html}{TLDP - Parameter Substitution}
    \pause
    \item \href{https://www.shell-tips.com/bash/math-arithmetic-calculation/}{Shell Tips - Doing Math in Bash}
    \pause
    \item See : \code{codes/variable-ex.sh}
\end{itemize}
\begingroup
\scriptsize
\begin{lstlisting}[backgroundcolor = \color{codegray}, language = Bash, showstringspaces=false]
# Simple variable assignment
$ V1="my simple string"
$ echo "V1 = ${V1}"   # worked!
$ echo "V1 = $V1"     # Note lack of braces, it'll hurt you someday
# Simple Math
$ V1=$(expr 3 + 1)
$ echo "V1 = ${V1}"   # worked! 
\end{lstlisting}
\endgroup
\end{frame}


\begin{frame}[fragile]
\frametitle{Conditionals}
Reference :
\begin{itemize}
    \item \href{https://tldp.org/LDP/Bash-Beginners-Guide/html/sect_07_01.html}{TLDP - If statements}
    \pause
    \item \code{codes/if-statement-ex.sh}
\end{itemize}
\begingroup
\scriptsize
\begin{lstlisting}[backgroundcolor = \color{codegray}, language = Bash, showstringspaces=false]
#### Simple if statement ####
echo "Simple if statement"
if [ -a codes/if-statement-ex.sh ]; then    
    echo "codes/if-statement-ex.sh EXISTS"
fi
#### Simple if else statement ####
echo "if / else statement"
if [ -a codes/if-statement-ex.sh ]; then   
    echo "codes/if-statement-ex.sh EXISTS"
else
    echo "codes/if-statement-ex.sh DNE"
fi
\end{lstlisting}
\endgroup
\end{frame}



\begin{frame}[fragile]
\frametitle{Loops}
Reference :
\begin{itemize}
    \item \href{https://tldp.org/LDP/abs/html/loops1.html}{TLDP - Loops }
    \pause
    \item \code{codes/loop-ex.sh}
\end{itemize}
\begingroup
\scriptsize
\begin{lstlisting}[backgroundcolor = \color{codegray}, language = Bash, showstringspaces=false]
#### Simple Counting for loop####
echo "Counting to 10..."
for((i=0; i<=10; i++)); do
    echo "i = ${i}"
done

echo "Looping over code files..."
for f in codes/*; do
    echo "f = ${f}"
done
\end{lstlisting}
\endgroup
\end{frame}


\begin{frame}[fragile]
\frametitle{Backticks - aka 'command substitution'}
Reference :
\begin{itemize}
    \item \href{https://tldp.org/LDP/abs/html/commandsub.html}{TLDP - Backticks and Command Substitution}
    \pause
    \item \code{codes/backticks-ex.sh}
\end{itemize}
\begingroup
\scriptsize
\begin{lstlisting}[backgroundcolor = \color{codegray}, language = Bash, showstringspaces=false, literate={`}{\textasciigrave}1,]
#### Simple command substitution ####
DIRSIZE=`du -hs`
echo "Size of current directory : ${DIRSIZE}"

#### In a Loop - sort of silly####
for f in `ls codes`; do
    echo "f = ${f}"
done
\end{lstlisting}
\endgroup
\end{frame}


%\begin{frame}[fragile]
%\frametitle{Primitive Error Handling}
%See : 
%\begingroup
%\scriptsize
%\begin{lstlisting}[backgroundcolor = \color{codegray}, language = Bash, showstringspaces=false]
%\end{lstlisting}
%\endgroup
%\end{frame}



\begin{frame}
\frametitle{Other Useful Features}
\begin{itemize}
    \item Pipes (revisit in a minute)
    \pause
    \item Stdout and stderr streams (ignored here)
    \pause
    \item Bash scripts (See \code{codes/})
\end{itemize}
\end{frame}


%\begin{frame}
%\frametitle{Pipes}
%Example : 
%\begin{itemize}
%    \item \code{ls | less}
%    \pause
%    \item Stdout and stderr streams (ignored here)
%    \pause
%\end{itemize}
%\end{frame}


%\begin{frame}
%\frametitle{Bash Scripts}
%Give examples here
%\begin{itemize}
%    \item 
%    \pause
%    \item
%    \pause
%\end{itemize}
%\end{frame}


\begin{frame}
\frametitle{Not Bash but still useful}
Not Strictly Bash, but incredibly useful 
\begin{itemize}
    \item \code{grep} - search text for strings
    \pause
    \item \code{find}   - Find files or directories
    \pause
    \item \code{man XXXX} - Get manual page
    \pause
    \item \code{less}, \code{tail}, \code{head} - for viewing files locally 
    \pause
    \item \code{gedit} - GUI text editor
    \pause
    \item \code{history} - get list of previous commands
\end{itemize}
\end{frame}


\begin{frame}[fragile]
\frametitle{Not Bash but still useful}
\begingroup
\scriptsize
\begin{lstlisting}[backgroundcolor = \color{codegray}, language = Bash, showstringspaces=false]
$ python codes/generate-fake-files.py   ## generate fake data
$ grep -rn "abc" fake-data/             ## search lines in files 
$ find fake-data -name "*0*" -print     ## Files with '0' in name
$ man find                              ## Read docs for find
$ less -SN fake-data/file_0_*           ## Read only file viewer
$ history                               ## print bash history 
$ history | tail -n 20                  ## Last 20 lines 
$ history | grep "find"                 ## Search bash history
\end{lstlisting}
\endgroup
\end{frame}





% Include slides on .bashrc / .profile
\begin{frame}
\frametitle{References}
\begin{itemize}
    \item \href{https://tldp.org/LDP/abs/html/parameter-substitution.html}{TLDP - Parameter Substitution}
    \bigskip
    \item \href{https://stackoverflow.com/questions/6697753/difference-between-single-and-double-quotes-in-bash}{Stackoverflow - Quotes explained in Bash }
    \bigskip
    \item \href{https://www.shell-tips.com/bash/math-arithmetic-calculation/}{Shell Tips - Doing Math in Bash}
    \bigskip
    \item \href{https://stackoverflow.com/a/5163260/4021436}{Stackoverflow - Dollar sign explained in Bash }
    \bigskip
    \item \href{https://tldp.org/LDP/Bash-Beginners-Guide/html/sect_07_01.html}{TLDP - If statements}
    \bigskip
    \item \href{https://tldp.org/LDP/abs/html/loops1.html}{TLDP - Loops }
\end{itemize}
\end{frame}





%%%%%%%%%% OTHER %%%%%%%%%%%%%
%%
%%% Include slides on .bashrc / .profile
%%\begin{frame}
%%\frametitle{.bashrc and .profile}
%%\code{.bashrc} and \code{.profile}
%%\bigskip
%%\begin{itemize}
%%    \item One (or both) run at the beginning of shell session.
%%    \bigskip
%%    \pause
%%    \item Can put \code{echo "Hello from X"} at top to determine which runs when you log in.
%%\end{itemize}
%%\end{frame}
%%
%%
%%% Include slides on .bashrc / .profile
%%\begin{frame}
%%\frametitle{.bashrc and .profile}
%%Can put things in it like:
%%\bigskip
%%\begin{itemize}
%%    \item Modules, e.g. \code{module load python/3.5.6}
%%    \bigskip
%%    \pause
%%    \item Set environmental variables, e.g. \code{module load python/3.5.6}
%%    \pause
%%    \bigskip
%%    \item Aliases, e.g. \code{alias ll='ls -l'}
%%    \pause
%%\end{itemize}
%%\end{frame}



\end{document}
